\documentclass[11pt]{article}

\usepackage{mathptmx}
\usepackage{url}
\usepackage{graphicx}
\usepackage{hyperref}
\graphicspath{{images/}}

\newcommand*{\p}[1]{\textup{\texttt{#1}}}
\newcommand*{\ls}{\textsc{LearnSAT}}
\newcommand*{\pl}{\textsc{Prolog}}
\newcommand*{\sw}{\textsc{SWI-Prolog}}
\newcommand*{\dt}{\textsc{dot}}

\textwidth=16cm
\textheight=22cm
\topmargin=0pt
\headheight=0pt
\oddsidemargin=5mm
\headsep=0pt
\renewcommand{\baselinestretch}{1.1}
\setlength{\parskip}{0.20\baselineskip plus 1pt minus 1pt}
\parindent=0pt

\begin{document}

\thispagestyle{empty}

\thispagestyle{empty}

\begin{center}

\textbf{\huge \ls\\\bigskip{}User's Guide and Software Documentation}

\bigskip
\bigskip

\textbf{\LARGE Moti Ben-Ari}

\bigskip
\bigskip

\large\url{http: //www.weizmann.ac.il/sci-tea/benari/}

\bigskip

\large Version 2.0


\end{center}

\vspace*{\fill}


\begin{center}
\copyright{} 2012-17 by Moti Ben-Ari.
\end{center}
This work is licensed under the Creative Commons Attribution-ShareAlike 3.0
License. To view a copy of this license, visit
\url{http://creativecommons.org/licenses/by-sa/3.0/}; or, (b) send a letter
to Creative Commons, 543 Howard Street, 5th Floor, San Francisco,
California, 94105, USA.

\bigskip\bigskip

 
\begin{center}
The following copyright notice applies to the programs described in this
document:\mbox{}\\\mbox{}\\
\copyright{} 2012-17 by Moti Ben-Ari.
\end{center}

This program is free software; you can redistribute it and/or
modify it under the terms of the GNU General Public License
as published by the Free Software Foundation; either version 3
of the License, or (at your option) any later version.
This program is distributed in the hope that it will be useful
but WITHOUT ANY WARRANTY; without even the implied warranty of
MERCHANTABILITY or FITNESS FOR A PARTICULAR PURPOSE.
See the GNU General Public License for more details.
You should have received a copy of the GNU General Public License
along with this program; if not, write to the Free Software
Foundation, Inc., 59 Temple Place - Suite 330, Boston, MA
02111-1307, USA.

\newpage

\section{Overview}

\ls{} is a program for learning about SAT solving. It implements the \emph{Davis-Putnam-Logemann-Loveland (DPLL)} algorithm with \emph{conflict-driven clause learning (CDCL)}, \emph{non-chronological backtracking (NCB)} and \emph{lookahead}. For a gentle introduction to SAT solvers, see \cite[Chapter~6]{mlcs}. The comprehensive reference is the \emph{Handbook of Satisfiability} \cite{SAT}.

The design of \ls{} is based on the following principles:

\begin{itemize}

\item A detailed trace of the algorithm's execution is
displayed. Its content can be set by the user.

\item Graphical representations of implication graphs and assignment
trees are generated automatically.

\item The implementation is in \pl{} so that the program will be concise and easy to understand.\\
Sections~\ref{s.module}--\ref{s.aux} document the software and the program has extensive comments. 

\item \ls{} is an open-source project.

\item The software is easy to install and use.

\item There is a \emph{tutorial} on SAT solving with \ls{} and many example programs are provided.
\end{itemize}

\section{Installation}

\ls{} is published in the \textit{Journal of Open Source Software}:
\begin{quote}
Ben-Ari, M. (2018) LearnSAT: A SAT Solver for Education.\\ \textit{Journal of Open Source Software}, 3(24), 639.\\\url{https://doi.org/10.21105/joss.00639}.
\end{quote}
The software itself can be found on GitHub:
\begin{center}
\url{https://github.com/motib/LearnSAT}.
\end{center}
See the file \texttt{readme.md} for instructions for installing and running the software, and for the community guidelines for working on the project.

There are three subdirectories: the \pl{} source code is in the directory \p{src}, programs are in the directory \p{examples} and the documentation is in the directory \p{docs}.

Download and install \sw{}:
\begin{center}
\url{http://www.swi-prolog.org/}.
\end{center}

The source files use the extension \p{pro} instead of the more usual
\p{pl} to avoid conflicts. During the installation
of \sw{}, if you associate the extension \p{pro} with \sw{}, a
program can be launched by double-clicking on its name in a file list. 

The directory \p{docs} contains the PDF and \LaTeX{} files for this document, an overview and the tutorial. It also includes a file \p{version.tex} to facilitate keeping the version numbers of the two documents consistent. The file \p{tikz.tex} contains the source code of diagrams that were generated using the Ti\textit{k}Z graphics package. The tutorial includes the generated PDF files (renamed for clarity) so that the it can be modified without installing the package.


\section{Running \ls}

The main module is in the file \p{dpll.pl}. It exports the following
predicates which are the only predicates that you need to run \ls{}:

\begin{center}
\begin{tabular}{|l|l|}
\hline
\multicolumn{2}{|c|}{\textbf{\large Predicates}}\\
\hline
\p{dpll}&Run the DPLL algorithm\\
\p{set\_display}&Set display options\\
\p{clear\_display}&Clear display options\\
\p{set\_mode}&Set the algorithmic mode\\
\p{set\_look}&Set the lookahead mode\\
\p{set\_decorate\_mode} & Set decoration as color or black-and-white\\ 
\p{set\_order}&Set the variable assignment order\\
\p{show\_config}&Show the current mode and display options\\
\p{usage}&Show the modes and display options \\
\hline
\end{tabular}
\end{center}

To check the satisfiability of a CNF formula, create a \pl{} program
that calls the predicate \p{dpll} with the clausal form of the formula
represented as a list of lists of literals. The predicate \p{use\_module} specifies the location of the \ls{} source code files. Alternatively, the \ls{} source files can be copied to the directory with the program.

The file \p{pigeon.pl} contains programs for the pigeonhole principle; here is the program for two holes and three pigeons, where \p{pij} means that pigeon \p{i} is in hole \p{j}:

\begin{verbatim}
:- use_module('../src/dpll').

hole2 :-
  dpll(
  [
  [p11, p12],   [p21, p22],   [p31, p32],   % Each pigeon in hole 1 or 2 
  [~p11, ~p21], [~p11, ~p31], [~p21, ~p31], % No pair is in hole 1
  [~p12, ~p22], [~p12, ~p32], [~p22, ~p32], % No pair is in hole 2
  ], _).
\end{verbatim}

The result (a satisfying assignment or \p{[]} if unsatisfiable) is
returned as the second argument, but can be left anonymous if only
the trace is of interest.

Once this file has been loaded (by double-clicking or by consulting
\p{[pigeon]}), the query:
\begin{verbatim}
?- hole2. 
\end{verbatim}
can be run. It reports that the clauses are unsatisfiable. After it terminates with \p{true}, press return to get a new prompt.

The output will be a trace of the DPLL algorithm:

\begin{verbatim}
1 ?- hole2.
LearnSAT (version 2.0)
Decision assignment: p11=0
Propagate unit:  p12 (p12=1) derived from: [p11,p12]
Propagate unit: ~p22 (p22=0) derived from: [~p12,~p22]
Propagate unit:  p21 (p21=1) derived from: [p21,p22]
Propagate unit: ~p31 (p31=0) derived from: [~p21,~p31]
Propagate unit:  p32 (p32=1) derived from: [p31,p32]
Conflict clause: [~p12,~p32]
Decision assignment: p11=1
Propagate unit: ~p21 (p21=0) derived from: [~p11,~p21]
Propagate unit:  p22 (p22=1) derived from: [p21,p22]
Propagate unit: ~p31 (p31=0) derived from: [~p11,~p31]
Propagate unit:  p32 (p32=1) derived from: [p31,p32]
Conflict clause: [~p22,~p32]
Unsatisfiable:
Statistics: clauses=9, variables=6, units=9, decisions=2, conflicts=2
true.
\end{verbatim}

The trace output can be directed to a file:

\begin{verbatim}
?- tell('hole2.txt'), hole2, told.
\end{verbatim}

%\newpage

\section{Controlling the algorithm}

There are predicates that set the mode of the algorithm and the display options. If you need to set a specific mode and set of display options when writing and experimenting with a program, you can write a predicate that you can run whenever \ls{} is initiated.

\begin{verbatim}
set_my_modes :-
  set_mode(ncb), 
  set_mode(look),
  set_display([default, dominator, dot, label, look]).
\end{verbatim}

\newpage

\subsection{Algorithmic mode}

\ls{} can run in one of three modes set by the predicate \p{set\_mode}:

\begin{center}
\begin{tabular}{|l|l|}
\hline
\p{dpll} & DPLL algorithm (default)\\
\p{cdcl} & DPLL with conflict-directed clause learning\\
\p{ncb} &  DPLL with CDCL and non-chronological backtracking\\
\hline
\end{tabular}
\end{center}

For the three-layer grid-pebbling problem, the statistics for the three modes are:
\begin{verbatim}
dpll: units=74, decisions=50, conflicts=26
cdcl: units=25, decisions=14, conflicts=8, learned clauses=5
ncb:  units=17, decisions=9,  conflicts=3, learned clauses=3
\end{verbatim}

The predicate \p{set\_look} is used to set whether lookahead is used or not:
\begin{center}
\begin{tabular}{|l|l|}
\hline
\p{none} & Disable lookahead (default)\\
\p{current} & Lookahead based on current clauses\\
\p{original} & Lookahead based on original clauses\\
\hline
\end{tabular}
\end{center}

With \p{current} lookahead enabled the statistics for the above problem are:
\begin{verbatim}
dpll: units=21, decisions=10, conflicts=6
cdcl: units=18, decisions=8,  conflicts=5, learned clauses=5
ncb:  units=16, decisions=6,  conflicts=3, learned clauses=3
\end{verbatim}
The lookahead algorithm is to choose the variable with the largest number of occurrences in the set of clauses resulting from evaluating them under the \p{current} partial assignment or in the \p{original} set of clauses. In both cases, learned clauses are taken into account.

\subsection{Order of the variables}

By default (and when lookahead is not enabled), decision assignments are made in lexicographical order of the atomic propositions. The order can be specified by using the predicate \p{set\_order}. The list argument to \p{set\_order} must be a permutation of the variables in the clauses. To restore the default order run \p{set\_order(default)}.

The \p{mlm} example finds a satisfying assignment \emph{without}
conflicts if the default order is changed:

\begin{verbatim}
set_order([x1,x2,x021,x031,x3,x4,x5,x6]).
\end{verbatim}

\newpage

\section{Display options}

\ls{} writes extensive trace output as it executes the algorithms. Each line starts with a string followed by a colon to make it easy to postprocess the output. The content of the trace is controlled using \p{set\_display} and \p{clear\_display}. The argument to these predicates can be \p{all} or \p{default}, or a single option or a list of options from Table~\ref{tab.display} (where * denotes the default options).

If \p{default} appears first in a list of display options, the others options will be added to the default ones.


\begin{table}
\begin{center}
\begin{tabular}{|l|l|}
\hline
%\multicolumn{2}{|c|}{\textbf{\large Display options}}\\
%\hline
\p{antecedent}  &  antecedents of the implied literals\\
\p{assignment}  &  assignments that caused a conflict\\
\p{backtrack} * &  level of non-chronological backtracking\\
\p{clause}      &  current set of clauses\\
\p{conflict} *  &  conflict clauses\\
\p{decision} *  &  decision assignments \\
\p{dominator}   &  computation of the dominator\\
\p{dot}         &  implication graphs (final) in dot format \\
\p{dot\_inc}    &  implication graphs (incremental) in dot format\\
\p{graph}       &  implication graphs (final) in textual format \\
\p{incremental} &  implication graphs (incremental) in textual format\\
\p{label}       &  dot graphs and trees labeled with clauses\\
\p{learned} *   &  learned clause by resolution\\
\p{look}        &  occurrences of variables for lookahead\\
\p{partial}     &  partial assignments so far \\
\p{resolvent} * &  resolvents created during CDCL \\
\p{result} *    &  result of the algorithm with statistics\\
\p{skipped} *   &  assignments skipped when backtracking \\
\p{sorted} *    &  assignments displayed in sorted order\\
\p{tree}        &  trees of assignments (final) in dot format\\
\p{tree\_inc}   &  trees of assignments (incremental) in dot format\\
\p{uip} *       &  unique implication points \\
\p{unit} *      &  unit clauses \\
\p{variables}   &  variables that are not assigned so far\\
\hline
\end{tabular}
\caption{Display options}\label{tab.display}
\end{center}
\end{table}

The display options depend on the algorithmic mode. If \p{dpll} is chosen, there are no learned clauses and the option \p{learn} has no effect. When lookahead is enabled, the display option \p{look} shows the number of occurrences of each variable when the clauses are evaluated under the current partial assignment. These clauses will be shown if the display option \p{partial} is enabled.

In \p{cdcl} and \p{ncb} modes, the assignments are written together with their levels in the format \p{p1=0@3}.

\p{antecedent} displays an implied assignment together with its
antecedent clause : \verb+p1@3/[~p1,p3]+.


\section{Configuration}

\p{config.pl} contains: the default modes, the default display options, and the \dt{} prologue and decorations. The predicate \p{decorate\_mode} determines which family of decorations will be used.

\p{show\_config} displays: the version and copyright notice, the
defaults from \p{config.pl}, changes from the default values, and the
variable order.

%\newpage

\section{Implication graphs}\label{s.impl}

When a conflict clause is encountered, \ls{} generates the implication
graph. Display option \p{graph} generates a textual representation,
while \p{dot} generates a graphical representation. Display option
\p{label} displays the antecedent clauses on each edge, not just their
numbers. Display option \p{dominator} emphasizes the following nodes:
the decision assignment at the current level, the dominator and
\p{kappa}.

\begin{center}
\includegraphics[keepaspectratio=true,width=\textwidth]{dom-color}
\end{center}

The graph can be displayed with color (default) or black-and-white
decoration:

\begin{verbatim}
set_decorate_mode(bw).
\end{verbatim}

The decorations are: decision assignments in red or bold; the decision assignment at the highest level, the \p{kappa} node and the dominator node with double borders; the cut with blue or dashed lines.

Display options \p{incremental} and \p{dot\_inc} generate graphs after
each step of the algorithm.

The graphics files in \dt{} format are rendered using
\textsc{GraphViz} (\url{http://www.graphviz.org/}):

\begin{verbatim}
dot -Tpng examples-ig-00.dot > examples-ig-00.png
\end{verbatim}

A script can be used to run \dt{} on all the generated files; in
Windows, the batch command is:

\begin{verbatim}
for %%F in (*.dot) do c:\"program files"\graphviz\bin\dot -Tpng %%F > %%~nF.png
\end{verbatim}

%\newpage

\section{Trees of assignments}

A tree showing the assignments to variables is generated by selecting
display option \p{tree} (next page). Trees are shown as directed acyclic
graphs whose nodes are assignments rather than variables. Decision nodes
are decorated with a red (or bold) border and conflict nodes with a
double red (or bold) border. A double green (or triple) border
indicates the node (if any) that causes the formula to become
satisfiable. The rendering of the \dt{} files is as described above for
the implication graphs.

\begin{center}
\includegraphics[keepaspectratio=true,height=.7\textheight]{tree1-color}
\end{center}

\section{DIMACS transformation}

The predicates in \p{dimacs.pl} convert a set of clauses in
\pl{} format (a list of lists of literals) to and from \emph{DIMACS cnf
format}:
\begin{itemize}
\item \p{to\_dimacs(File, Comment, Clauses)} converts the \pl{}
\p{Clauses} into DIMACS format and writes them to the \p{File} with the
\p{Comment}.
\item \p{from\_dimacs(Predicate, InFile, OutFile)} reads \p{InFile} in
DIMACS format and writes a \pl{} program to \p{OutFile} as
\verb+Predicate :- dpll(List, _).+
\end{itemize}


\section{Module structure}\label{s.module}

The \ls{} software consists of the following source files (not including the test programs and the program for DIMACS conversion):

\begin{itemize}
\item \p{dpll.pl}: Main module.

\item \p{cdcl.pl}: Algorithms for CDCL and the implication graph.

\item \p{auxpred.pl}: Auxiliary predicates for the algorithms. 

\item \p{io.pl}: Predicates for writing assignments, clauses and
implication graphs.

\item \p{display.pl}: Display the trace using predicates \p{display/n},
where the first argument is a display option and the additional
arguments supply the data to be displayed.

\item \p{config.pl}: Default configuration data.

\item \p{counters.pl}: Maintains counters for the number of clauses, variables, units, decisions and conflicts. These are used for printing the statistics. Defines counters for adding a number to the file names for implication graphs and trees of assignments.

\item \p{modes.pl}: Sets, clears and checks the modes and the display options. Implements \p{usage} and \p{show\_config}. The dummy display option \p{none} is used to distinguish between the initial state (no options, so set the default options) and a state where all options have been cleared.

\item \p{dot.pl}: Generate the \dt{} files of the implication graphs
and the assignment trees.
\end{itemize}

\section{Data structures}

The following dynamic predicates are used:
\begin{itemize}

\item In \p{dpll.pl}: The counts of variable \p{occurrences} when \p{original} lookahead is enabled.

\item In \p{cdcl.pl}: The non-chronological backtracking level in the \p{backtrack}.

\item In \p{cdcl.pl}: The learned clauses are stored as a list in \p{learned}.

\item In \p{auxpred.pl}: If the user specifies the order of assignment
to variables, the list is stored in \p{variables\_list}.

\item In \p{dot.pl}: \p{node} and \p{edge} are used when building the assignment trees. They store the paths leading to previous conflict nodes and are updated with each new path.

\item In \p{mode.pl}: \p{alg\_mode}, \p{look\_mode},\p{decorate\_mode}, \p{display\_option}.

\item In \p{counters.pl}: Counters for reporting statistics and for generating \p{dot} file names.
\end{itemize}

Two functors are used to create terms:
\begin{itemize}

\item \p{assign} represents an assignment and takes four arguments: a
variable, its value, its level and either \p{yes} if this is a decision
assignment or the antecedent clause of an implied assignment.

\item \p{graph} is an implication graph. Its arguments are a list of
nodes which are assignments and a list of edges which are terms with
functor \p{edge} and three arguments: the source and target nodes
and the clause that labels the edge.
\end{itemize}

The initial list of clauses is retained and completely re-evaluated as assignments are added. This is not efficient but it facilitates the display of the trace of the algorithms.


\section{The DPLL algorithm}

\p{dpll/2} implements the DPLL algorithm on a set of
clauses represented as a list of lists of literals. It returns a list of
satisfying assignments or the empty list if the clauses are
unsatisfiable. As part of its initialization, the set of variables in
the clauses is obtained from the list.

\p{dpll/2} invokes \p{dpll/6} which is the main recursive
predicate for performing the algorithm. If the set of variables to be
assigned to is empty, the set of clauses is satisfiable. Otherwise,
\p{dpll/6} tries to perform unit propagation by searching for a unit and
then evaluating the set of clauses. When no more units remain, it
chooses a decision assignment and evaluates the set of clauses.

\p{ok\_or\_conflict} is called with the result of the
evaluation of unit propagation or the decision assignment. If the
result is not a conflict, the variable chosen is deleted and \p{dpll/6}
is called recursively. If there was a conflict, the implication graph is
constructed and a learned clause is generated from the graph; then
\p{ok\_or\_conflict} fails so that backtracking can try a new
assignment.

\p{evaluate} receives a set of clauses and evaluates them,
returning \p{ok} or \p{conflict}. For each clause it calls
\p{evaluate\_clause}, which returns \p{satisfied}, \p{unsatisfied},
\p{unit} or \p{not\_resolved}.

\p{find\_unit} calls \p{evaluate\_clause} on each clause and succeeds if it returns \p{unit}.

\p{choose\_assignment} returns a decision assignment. The decision variable is the first unassigned variables unless lookahead is enabled (Section~\ref{s.look}). \p{choose\_value} assigns first 0 and then 1. A conditional construct with an internal cut-fail implements non-chronological backtracing. (See Section~4.7 of the \sw{} Reference Manual.)

\section{Lookahead}\label{s.look}

The predicates for lookahead appear in \p{dpll.pl}. \p{choose\_variable\_current} takes the set of clauses and the current assignment and returns the chosen variable. First the clauses are evaluated under the current assignment; then the clauses are flattened into a single list of literals, which is transformed to a list of variables. The variable with the largest number of occurrences is returned. The number of occurrences of a variable is maintained as pair \p{Count-Variable}, which is sorted in descending order of \p{Count}. The list of pairs is computed by the predicate \p{count\_occurrences}.

For lookahead using the original set of clauses, \p{set\_original\_lookahead} is called during initialization to create the database \p{occurrences}. \p{choose\_variable\_original} finds the variable with the largest number of occurrences that has not yet been assigned. \p{update\_variable\_occurrences} is called when a clause has been learned to modify \p{occurrences}.


\section{CDCL and NCB}

The implication graph is built incrementally. Whenever a unit clause is
found, \p{extend\_graph} is called with the unit clause, its number (to
label the new edges), the assignment it implies (to create the new
target of the edges) and the graph constructed so far. For each literal
(except the one implied), a new edge is created and when the list of
literals has been traversed, the new node is created. When a conflict is
encountered, the \p{kappa} node and its incoming edges are added.
Two predicates for computing a learned clause are now called.

\p{compute\_learned\_clause\_by\_resolution} starts with the conflict
clause (the antecedent clause of the \p{kappa} node) as the current
clause. \p{learn\_clause\_from\_antecedents} uses the list of
assignments so far to locate the antecedent of the assignment associated
with the current clause. \p{resolve} resolves the two clauses and
\p{learn\_clause\_from\_antecedents} is called with the resolvent as the
new current clause. The algorithm terminates when \p{check\_uip}
identifies the current clause as a UIP and this clause becomes the
learned clause. The clause learned by resolution is added to the list of learned clauses in the dynamic predicate \p{learned}.

\p{compute\_learned\_clause\_by\_dominator} locates a dominator of the
highest-level decision assignment node relative to the \p{kappa} node.
\p{get\_paths\_to\_kappa} returns a list of all paths
from the decision node to the \p{kappa} node. This list is an argument
to \p{get\_dominator} which finds a node that appears on all the paths.
\p{get\_paths\_to\_kappa} is called again to find paths from decision
assignments at \textit{lower} levels to \p{kappa} and then paths that go
through the dominator are removed. From the remaining paths, edges are
located that go from nodes of lower level to nodes of the highest level.
The complements of the assignments at the source nodes of these edges,
together with the complement of the assignment at the dominator, define
the learned clause.

\p{compute\_backtrack\_level} returns the highest level of an assignment
in the learned clause except for the current level.

\textbf{Open issue} After learning a clause, the computation backtracks to choose a new decision assignment. However, the learned clause can cause a variable to be assigned and this should be handled before the new decision assignment. In the \p{mlm} example presented in the tutorial:
\begin{verbatim}
Learned clause from resolution: [x021,~x4]
Decision assignment: x1=1@3
Propagate unit: ~x4 (x4=0@3) derived from: [x021,~x4]
\end{verbatim}
since \p{x021} was assigned 0 at level 1, unit propagation of the learned clause \verb+[x021,~x4]+ will cause \p{x4} to be assigned \emph{at the same level}: \p{x4=0@1}, not \p{x4=0@3} resulting from the new decision assignment to \p{x1}. This issue can affect the computation of subsequent learned clauses.

\section{Auxiliary predicates}\label{s.aux}

\begin{itemize}

\item \p{is\_assigned} checks if a literal has been assigned a value
and if so returns that value.

\item \p{get\_variables\_of\_clauses} gets the list of occurrences of variables from a set of clauses (for lookahead) and the list of variables with duplicates removed (for making decision assignments).

\item \p{literals\_to\_variables} takes a list of literals and returns a
sorted set of the variables corresponding to the literals.

\item Transformations of variables, literals, assignments and clauses: \p{to\_variable}, \p{to\_complement}, \p{to\_assignment}, \p{to\_literal}, \p{to\_complemented\_clause}.

\item \p{set\_order} changes the order of the variables for decision assignments. \p{get\_order} used to display the order. \p{order\_variables} returns the ordered list of variables if one has been set.

\end{itemize}

%\newpage

\bibliographystyle{plainurl}
\bibliography{learnsat}
\end{document}
